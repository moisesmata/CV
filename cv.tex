%-----------------------------------------------------------------------------------------------------------------------------------------------%
%	The MIT License (MIT)
%
%	Copyright (c) 2021 Jitin Nair
%
%	Permission is hereby granted, free of charge, to any person obtaining a copy
%	of this software and associated documentation files (the "Software"), to deal
%	in the Software without restriction, including without limitation the rights
%	to use, copy, modify, merge, publish, distribute, sublicense, and/or sell
%	copies of the Software, and to permit persons to whom the Software is
%	furnished to do so, subject to the following conditions:
%	
%	THE SOFTWARE IS PROVIDED "AS IS", WITHOUT WARRANTY OF ANY KIND, EXPRESS OR
%	IMPLIED, INCLUDING BUT NOT LIMITED TO THE WARRANTIES OF MERCHANTABILITY,
%	FITNESS FOR A PARTICULAR PURPOSE AND NONINFRINGEMENT. IN NO EVENT SHALL THE
%	AUTHORS OR COPYRIGHT HOLDERS BE LIABLE FOR ANY CLAIM, DAMAGES OR OTHER
%	LIABILITY, WHETHER IN AN ACTION OF CONTRACT, TORT OR OTHERWISE, ARISING FROM,
%	OUT OF OR IN CONNECTION WITH THE SOFTWARE OR THE USE OR OTHER DEALINGS IN
%	THE SOFTWARE.
%	
%
%-----------------------------------------------------------------------------------------------------------------------------------------------%

%----------------------------------------------------------------------------------------
%	DOCUMENT DEFINITION
%----------------------------------------------------------------------------------------

% article class because we want to fully customize the page and not use a cv template
\documentclass[a4paper,12pt]{article}

%----------------------------------------------------------------------------------------
%	FONT
%----------------------------------------------------------------------------------------

% % fontspec allows you to use TTF/OTF fonts directly
% \usepackage{fontspec}
% \defaultfontfeatures{Ligatures=TeX}

% % modified for ShareLaTeX use
% \setmainfont[
% SmallCapsFont = Fontin-SmallCaps.otf,
% BoldFont = Fontin-Bold.otf,
% ItalicFont = Fontin-Italic.otf
% ]
% {Fontin.otf}

%----------------------------------------------------------------------------------------
%	PACKAGES
%----------------------------------------------------------------------------------------
\usepackage{url}
\usepackage{parskip} 	

%other packages for formatting
\RequirePackage{color}
\RequirePackage{graphicx}
\usepackage[usenames,dvipsnames]{xcolor}
\usepackage[scale=0.9]{geometry}

%tabularx environment
\usepackage{tabularx}

%for lists within experience section
\usepackage{enumitem}

% centered version of 'X' col. type
\newcolumntype{C}{>{\centering\arraybackslash}X} 

%to prevent spillover of tabular into next pages
\usepackage{supertabular}
\usepackage{tabularx}
\newlength{\fullcollw}
\setlength{\fullcollw}{0.47\textwidth}

%custom \section
\usepackage{titlesec}				
\usepackage{multicol}
\usepackage{multirow}

%CV Sections inspired by: 
%http://stefano.italians.nl/archives/26
\titleformat{\section}{\Large\scshape\raggedright}{}{0em}{}[\titlerule]
\titlespacing{\section}{0pt}{10pt}{10pt}

%for publications
\usepackage[style=authoryear,sorting=ynt,maxbibnames=99]{biblatex}

%Setup hyperref package, and colours for links
\usepackage[unicode, draft=false]{hyperref}
\definecolor{linkcolour}{rgb}{0,0.2,0.6}
\hypersetup{colorlinks,breaklinks,urlcolor=linkcolour,linkcolor=linkcolour}
\addbibresource{citations.bib}
\setlength\bibitemsep{1em}

%for social icons
\usepackage{fontawesome5}

%debug page outer frames
%\usepackage{showframe}


% job listing environments
\newenvironment{jobshort}[2]
    {
    \begin{tabularx}{\linewidth}{@{}l X r@{}}
    \textbf{#1} & \hfill &  #2 \\[3.75pt]
    \end{tabularx}
    }
    {
    }

\newenvironment{joblong}[4]
    {
    \noindent\textbf{#1} \dotfill #2 \\
    \noindent #3 \dotfill #4 \\[3.75pt]
    \begin{minipage}[t]{\linewidth}
    \begin{itemize}[nosep,after=\strut, leftmargin=1em, itemsep=3pt,label=--]
    }
    {
    \end{itemize}
    \end{minipage}    
    }

\newenvironment{joblongsameorg}[2]
    {
    \noindent #1 \dotfill #2 \\[3.75pt]
    \begin{minipage}[t]{\linewidth}
    \begin{itemize}[nosep,after=\strut, leftmargin=1em, itemsep=3pt,label=--]
    }
    {
    \end{itemize}
    \end{minipage}    
    }



%----------------------------------------------------------------------------------------
%	BEGIN DOCUMENT
%----------------------------------------------------------------------------------------
\begin{document}

% non-numbered pages
\pagestyle{empty} 

%----------------------------------------------------------------------------------------
%	TITLE
%----------------------------------------------------------------------------------------

% \begin{tabularx}{\linewidth}{ @{}X X@{} }
% \huge{Your Name}\vspace{2pt} & \hfill \emoji{incoming-envelope} email@email.com \\
% \raisebox{-0.05\height}\faGithub\ username \ | \
% \raisebox{-0.00\height}\faLinkedin\ username \ | \ \raisebox{-0.05\height}\faGlobe \ mysite.com  & \hfill \emoji{calling} number
% \end{tabularx}

\begin{tabularx}{\linewidth}{@{} C @{}}
\Huge{Moises Mata} \\[7.5pt]
\href{https://github.com/moisesmata}{\raisebox{-0.05\height}\faGithub\ moisesmata} \ $|$ \ 
\href{https://linkedin.com/in/moises-mata-t}{\raisebox{-0.05\height}\faLinkedin\ moisesmata} \ $|$ \ 
\href{https://moises-mata.com}{\raisebox{-0.05\height}\faGlobe \ moises-mata.com} \ $|$ \ 
\href{mailto:mm6155@columbia.edu}{\raisebox{-0.05\height}\faEnvelope \ mm6155@columbia.edu} \\
\end{tabularx}

%----------------------------------------------------------------------------------------
%	EDUCATION
%----------------------------------------------------------------------------------------
\section{Education}
\noindent Columbia University, BA in Computer Science and Applied Mathematics \dotfill Sept 2022 -- May 2026
%----------------------------------------------------------------------------------------
%	PUBLICATIONS
%----------------------------------------------------------------------------------------
\section{Publications}
\begin{refsection}[citations.bib]
\nocite{*}
\printbibliography[heading=none]
\end{refsection}

%----------------------------------------------------------------------------------------
% EXPERIENCE SECTIONS
%----------------------------------------------------------------------------------------

%Experience
\section{Research \& Work Experience}

\begin{joblong}{Barnard/Dartmouth Accessible and Accelerated Robotics Lab}{New York, NY}{Undergraduate Researcher}{Jan 2025 -- Present}
\item Diagnosing hardware constraints and modernizing the \textbf{C++} codebase for \href{https://tinympc.org/}{TinyMPC}, an open-source convex Model Predictive Control solver designed for resource-constrained platforms. Advising professor: Brian Plancher.
\item Implementing \textbf{meta-learning} algorithms to enable online controller optimization and adaptation across systems with different dynamics.
\end{joblong}

\begin{joblong}{Columbia SNL Lab}{New York, NY}{Undergraduate Researcher}{Sept 2025 -- Present}
\item Developing a high-assurance, real-time runtime for spacecraft flight software using \textbf{eBPF}, \textbf{F Prime}, and \textbf{C/C++}, enabling dynamic multithreaded execution of mission logic with strict safety and millisecond deadline guarantees. Advising Professor: Junfeng Yang. (\textbf{Publication In Progress}) 
\end{joblong}

\begin{joblong}{NASA Jet Propulsion Lab, Small Scale Flight Software Group}{Pasadena, CA}{Intern, Flight Software Engineering}{May 2025 -- Aug 2025}
\item Developed core infrastructure for the \href{https://fprime.jpl.nasa.gov/}{F Prime} flight software framework (\href{https://github.com/nasa/fprime/releases/tag/v4.0.0}{v4.0.0 release}).
\item Architected and implemented reusable software stacks (Communications, Command and Data Handling, Filesystem Support, Data Products), reducing setup time for new missions.
\end{joblong}

\begin{joblong}{NASA Jet Propulsion Lab, Exoplanet Discovery \& Science}{Pasadena, CA}{Undergraduate Research Fellow}{May 2024 -- Aug 2024}
\item Trained binary classification models using \textbf{Python}, \textbf{scikit-learn}, within \textbf{Jupyter Notebook} to predict the presence of habitable planets from simulated observational data.
\item Collaborated with Dr. Yasuhiro Hasegawa to apply models to Kepler multi-planet systems. (\textbf{Publication In Progress})
\end{joblong}

\begin{joblong}{Columbia Center for Student Advising}{New York, NY}{Tutor, Calculus I and Digital Systems}{Sept 2023 -- Sept 2024}
\item Tutor for Calculus I and Fundamentals of Computer Systems (CSEE 3827); tracked student progress and prepared targeted material for class and exams.
\end{joblong}

\begin{joblong}{Columbia Astronomy}{New York, NY}{Undergraduate Researcher}{Dec 2022 -- Jul 2023}
\item Collaborated with Professor Kathryn Johnston to study Milky Way stellar streams using \textbf{Gaia} and \textbf{Pan-STARRS} data.
\item Analyzed the substructure of the Ophiuchus stream and simulated initial formation conditions.
\end{joblong}

%----------------------------------------------------------------------------------------
%	LEADERSHIP EXPERIENCE
%----------------------------------------------------------------------------------------
\section{Leadership Experience}

\begin{joblong}{Columbia Space Initiative}{New York, NY}{Executive Board, Co-President}{March 2025 -- Present}
\item Leading \textbf{Columbia's largest engineering club} (\textbf{250+ active members}, \textbf{\$200k annual budget}) as \textbf{Co-President}, coordinating strategy across \textbf{13 active projects} and managing overall budget allocation.
\item Organizing major events including company visits, faculty talks, and astronaut visits, engaging Columbia and the broader NYC community in aerospace.
\item Conducting career development workshops (resume/CV, internships) for Columbia's undergraduate engineering population and leading external outreach to \textbf{1000+ middle and high school students}, primarily translating engineering concepts into Spanish to inspire underrepresented demographics.
\end{joblong}

\begin{joblong}{Columbia Space Initiative}{New York, NY}{Executive Board, Treasurer}{March 2024 -- March 2025}
\item Administered and operated the club's \textbf{\$200k annual budget}, managing procurement, orders, and financial coordination across \textbf{13 projects} and the Mechanical Engineering department.
\end{joblong}

%----------------------------------------------------------------------------------------
%	PROJECTS
%----------------------------------------------------------------------------------------
\section{Projects}

\begin{joblong}{Columbia Space Initiative}{New York, NY}{Columbia Flight Software Lead, \href{https://docs.proveskit.space/en/latest/}{PROVES} Alcyone (LionCub)}{Aug 2024 -- Present}
\item 1U CubeSat on a manifested launch for April 2026. Will be \textbf{Columbia's first satellite}, with a mission to take pictures of the Earth from space. Part of the larger PROVES project.
\item Implemented critical \textbf{F Prime} components including hardware watchdog and power load switch management. Developed camera control software with \textbf{UART} communication for image capture and onboard storage, enabling Columbia's first satellite imaging capabilities.
\item Mentored younger members on flight software concepts and best practices.
\end{joblong}

\begin{joblong}{Columbia Space Initiative}{New York, NY}{\href{https://www.nasa.gov/learning-resources/spacesuit-user-interface-technologies-for-students/}{NASA SUITS} Mission Co-Lead}{Sept 2023 -- Jul 2024}
\item Co-authored a proposal selected as one of \textbf{17 national finalists} and led testing activities at \textbf{NASA Johnson Space Center}.
\item Developed an AR interface in \textbf{Unity} (\textbf{C\#}) for astronaut assistance, deployed on \textbf{Microsoft HoloLens 2}. Testers successfully completed a simulated Mars mission on-site.
\end{joblong}

%----------------------------------------------------------------------------------------
%	SKILLS
%----------------------------------------------------------------------------------------
\section{Skills}
\begin{tabularx}{\linewidth}{@{}l X@{}}
Programming & \normalsize{C, C++, C\#, Python, CircuitPython, MATLAB, F\textsuperscript{\'{}}/F Prime, Unity, scikit-learn, Jupyter}\\
Domains & \normalsize{Robotic Optimization and Control, Embedded Systems, Flight Software}\\
Languages & \normalsize{English, Spanish (native proficiency), French (limited)}\\
Certifications & \normalsize{Amateur Radio (Technician Class)}\\
\end{tabularx}

\vfill
\center{\footnotesize Last updated: \today}

\end{document}
